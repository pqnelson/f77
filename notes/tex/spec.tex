\section{Specification}
We will now describe the grammar for our particular subset of \FORTRAN/~77.
This is similar to that found in the first chapter of van Loan and
Coleman~\cite{vanloan1987handbook}. The goal will be that any program
which our compiler accepts will also be compilable by any other \FORTRAN/~77 compiler.
Like most imperative programming languages, we can partition the
language into statements and expressions.

\subsection{Important differences}
We intentionally make a number of design decisions which deviates from
the \FORTRAN/~77 Standard:
\begin{enumerate}
\item\textbf{No implicit parameters.} If an identifier does not appear
  in the declarations area of the program unit, or names a function or
  subroutine, then an error is raised.
\item \textbf{Reserved keywords.} Unlike \FORTRAN/ we have reserved
  keywords, which are NOT CASE SENSITIVE, namely:
  \begin{itemize}
  \item statement related: \texttt{continue}, \texttt{do}, \texttt{else}, \texttt{endif}, \texttt{end}, \texttt{function}, \texttt{goto}, \texttt{if}, \texttt{program}, \texttt{return}, \texttt{stop}, \texttt{subroutine}, \texttt{then}
  \item Primitive functions provided by the compiler: \texttt{read}, \texttt{write}
  \item logical constants: \texttt{.true.}, \texttt{.false.}
  \item relational operators: \texttt{.le.}, \texttt{.lt.}, \texttt{.ge.}, \texttt{.gt.}, \texttt{.eq.}, \texttt{.ne.}
  \item logical connectives: \texttt{.and.}, \texttt{.or.}, \texttt{.not.}, \texttt{.eqv.}, \texttt{.neqv.}
  \item types: \texttt{integer}, \texttt{real}, \texttt{character}, \texttt{logical}
  \end{itemize}
\item\textbf{Array sections.} This was found in \FORTRAN/~90, and was
  available only for character arrays in \FORTRAN/~77. But it was easier
  to just implement it for \emph{all} arrays.
\end{enumerate}

\subsection{Program structure}
The program consists of a \texttt{program\dots end} block. The grammar
looks like:

\begin{quotation}
\noindent\texttt{program} $\langle$\textit{name\/}$\rangle$

$\langle$\textit{declarations\/}$\rangle$

$\langle$\textit{statements\/}$\rangle$

\noindent\texttt{stop}

\noindent\texttt{end}
\end{quotation}

The \FORTRAN/~77 Standard is quite clear that the \texttt{program}
statement must be the first ``program unit'' of a main program (\S3.5),
followed by functions and subroutines.

\subsection{Line length}
Each line in a \FORTRAN/~77 code must conform to certain column position
rules:
\begin{itemize}
\item Column 1 --- must be blank unless the line is a comment or a label;
\item Columns 1--5 --- Statement label which, if present, must be a
  positive integer (optional);
\item Column 6 --- indicates continuation of previous line (optional),
  the continuation mark may be any character, but usually is a plus
  sign, an ampersand, or a digit [the informal practice uses
  ``2'' for the second line, ``3'' for the third line, etc.];
\item Columns 7--72 --- The \FORTRAN/ statement;
\item Columns 73--80 --- Sequence number (optional)
\end{itemize}
The basic algorithm for determining what to do with a line:
\begin{enumerate}
\item Is the character in column 1 either ``\texttt{C}'' or
  ``\texttt{*}''? If so, it is a comment, skip the rest of the
  line. Otherwise, go to the next step.
\item Are any characters in column 1 through 5 a nonzero digit? If so,
  discard the leading zeros, and the number formed from these digits is
  the label for a new statement. Otherwise, go to the next step.
\item Is the character in column 6 not ``\texttt{0}'' or blank? If so,
  then this line is a continuation of the previous line. Otherwise go to
  the next step.
\item The line is a new statement.
\end{enumerate}

\subsubsection{Labels}
Labels must be nonzero positive integers no longer than 5 digits. Labels
may have leading zeros, but these are discarded (so ``\texttt{010}'' and
``\texttt{10}'' are treated as identical labels).

\subsection{Statements}
The statements which we will support include:
\begin{itemize}
\item input/output statements,
\item continue statements,
\item do-loops,
\item if statements,
\item goto statements,
\item assignment statements.
\end{itemize}
Further, \emph{after} the \texttt{end} of the program block, we allow
declaration of functions and subroutines.

\subsubsection{Input, output}
We will have \texttt{write(*,*)} take a comma-separated list of
expressions and write them to the screen. Dually, we have
\texttt{read(*,*)} take a list of variables, and read values from the
keyboard then assign them to the variables.

\subsubsection{Continue statements}
This is the same as the ``skip'' statement in Hoare logic, a
``\texttt{continue}'' does literally nothing. (This is what the Standard
says as well, see \S11.11 of the \FORTRAN/ 77 Standard.)

\subsubsection{Loops}
The semantics for loops in \FORTRAN/~77 is quite different than in
C-like languages. The syntax looks quite simple:
\begin{quotation}
\noindent\texttt{do} $\langle$\textit{label\/}$\rangle$
$\langle$\textit{variable\/}$\rangle$
\texttt{=} $\langle$\textit{initial value\/}$\rangle$\texttt{,}
$\langle$\textit{final value\/}$\rangle$\texttt{,}
$\langle$\textit{optional step\/}$\rangle$

$\langle$\textit{statements\/}$\rangle$

\noindent$\langle$\textit{label\/}$\rangle$ \texttt{continue}
\end{quotation}
Here the loop variable must be an integer; the initial value, final
value, and optional step may be any arithmetic expression which
evaluates to an integer value.

\FORTRAN/ 77 executes ``\texttt{do $\ell$ i = $m_{1}$, $m_{2}$, $m_{3}$}''
in the following steps:
\begin{enumerate}
\item Evaluate $m_{1}$ to produce the value $v_{1}$, $m_{2}$ to produce
  the value $v_{2}$, and $m_{3}$ to the value $v_{3}$ (or, if absent,
  defaulting to $1$); the iteration
  count is set to $\max((v_{2}-v_{1})/v_{3}, 0)$.
\item Initialize the loop variable $i$ to its initial value. If the
  iteration count is nonzero, continue with the next step; otherwise,
  goto the label $\ell$.
\item\label{do-loop:step2} Execute the body of the loop, which is
  determined as the statements up to and including the statement with
  the given label;
\item Add to the loop variable the step amount (defaulting to 1 if omitted);
\item Test for termination: if the step amount is negative, check the
  loop variable is less than or equal to the stopping value; if the step
  amount is positive, check the loop variable is greater than or equal
  to the stopping value. In these cases, we go to the line with the
  specified label, which is idiomatically a \texttt{continue} line;
  otherwise, we return to Step~\ref{do-loop:step2}.
\end{enumerate}
Some commentaries state that the do-loop's body is always executed at least
once, but I am unable to adequately confirm this behaviour (I don't have
access to a \FORTRAN/ 77 compiler and \FORTRAN/ 90 changed this
behaviour: \S8.1.4.4.2 of the \FORTRAN/
90 Standard states that a do-loop's body is executed only after the
iteration count is tested).
This is an overly simplified version of \FORTRAN/~77's
do-loop.\footnote{For the gory details, see \FORTRAN/ 77 standard 11.10,
e.g., \url{https://wg5-fortran.org/ARCHIVE/Fortran77.html}}

\textbf{Restrictions} on the do-loop:
\begin{enumerate}
\item \textbf{Caution:} the loop variable must never be changed by other
statements in the loop's body. (\S11.10.5 of the Standard)
\item The target label must be inside the same programming unit
  [program, function, or subroutine]. (\S11.10.2 of the Standard)
\item If a do-loop appears within another do-loop, the range of the
  do-loop specified by the inner [nested] loop must be contained
  entirely within the range of the outer [parent] loop. Similarly, if a
  do-loop appears in an if-block (or any block statement), then it must
  be contained entirely within that block. (\S11.10.2 of the Standard)
\item We impose the nonstandard [but idiomatic] restriction that
  do-loops work with integer parameters.
\item Transfer of control [by, e.g., a \texttt{goto} or as the terminal
  of a do-loop] into the body of a do-loop is not permitted. (\S11.10.8)
\end{enumerate}

\subsubsection{If Statements}
We have several forms of if-statements:
\begin{quote}
\noindent\texttt{if (}$\langle$\textit{logical expr\/}$\rangle$\texttt{)} $\langle$\textit{statement\/}$\rangle$
\end{quote}
This must be written on one single line. If more than one statement
needs to be executed, we have to write:
\begin{quotation}
\noindent\texttt{if (}$\langle$\textit{logical expr\/}$\rangle$\texttt{)} \texttt{then}

$\langle$\textit{statements\/}$\rangle$

\noindent\texttt{endif}
\end{quotation}
More generally, we may have:
\begin{quotation}
\noindent\texttt{if (}$\langle$\textit{logical expr\/}$\rangle$\texttt{)} \texttt{then}

$\langle$\textit{statements\/}$\rangle$

\noindent\texttt{else if (}$\langle$\textit{logical expr\/}$\rangle$\texttt{)} \texttt{then}

$\langle$\textit{statements\/}$\rangle$

\dots

\noindent\texttt{else}

$\langle$\textit{statements\/}$\rangle$

\noindent\texttt{endif}
\end{quotation}
The execution flow is from top to bottom.

\subsubsection{Goto Statements}
Every statement has an optional label, and we can jump to a statement
with a label by using the ``\texttt{goto}'' statement. Its grammar:
\begin{quote}
\texttt{goto} $\langle$\textit{label\/}$\rangle$
\end{quote}
This can be combined with ``\texttt{if}'' to form loops, for example:
\begin{lstlisting}
      integer n
      n = 1
10    if (n .le. 100) then
          write (*, *) n
          n = 2*n
          goto 10
      endif
\end{lstlisting}
This is the equivalent of a ``while-loop'' in C. Similarly, a ``do-while''
\begin{lstlisting}
label continue
C     Statements
      if (logicalExpression) goto label
\end{lstlisting}

\textbf{Restrictions} on \texttt{goto} statements:
\begin{enumerate}
\item The label must be within the same programming unit [program,
  function, or subroutine] (\S11.1)
\item We restrict focus to the uncomputed goto statement, disallowing
  computed goto statements (\S11.2 disallowed) and assigned goto
  statements (\S11.3 disallowed)
\item We stress we cannot have goto jump into the middle of a do-loop.
\end{enumerate}

\subsubsection{Assignment Statements}
Assignments have the form:
\begin{quote}
$\langle$\textit{variable\/}$\rangle$ \texttt{=} $\langle$\textit{expression\/}$\rangle$
\end{quote}

\subsection{Declarations}
The declaration area consists of a sequence of declaration
statements. This looks like:
\begin{quote}
$\langle$\textit{type\/}$\rangle$ $\langle$\textit{comma-separated list of identifers\/}$\rangle$
\end{quote}
We can have multiple declaration statements of the same type. For
example,
\begin{lstlisting}
      program circle2
      real r
      real area
      real pi
      parameter (pi = 3.14159)

C This program reads a real number r and prints
C the area of a circle with radius r.

      write (*, *) 'Give radius r:'
      read  (*, *) r
      area = pi * r * r
      write (*, *) 'Area = ', area

      stop
      end
\end{lstlisting}
\subsubsection{Identifiers}
Identifiers in \FORTRAN/ must involve no more than six characters chosen
from the alphanumeric set \texttt{ABC\dots XYZ0123456789}, and must
begin with a letter.

\subsubsection{Types}
We only really need 4 types, though \FORTRAN/~77 had 6 types. The types
we support: \texttt{integer}, \texttt{real} (for floating-point
numbers), \texttt{logical} (for Boolean values), and \texttt{character}.

Note that \texttt{real} refers to single precision floating-point
numbers, \texttt{real*8} refers to double precision floating-point
numbers, and \texttt{real*16} refers to quad precision floating-point
numbers [if supported].

\subsubsection{Arrays}
We declare arrays by adding its dimension to the variable name, e.g.,
\begin{lstlisting}
      real a(20)
      real*64 A(3, 5)
\end{lstlisting}
where multi-dimensional arrays simply add the dimensions with commas. It
is stored in memory as the sequence: $A(1,1)$, $A(1,2)$, \dots,
$A(1,5)$, $A(2,1)$, \dots, $A(2,5)$, $A(3,1)$, $A(3,2)$, \dots, $A(3,5)$.
For locality, you would want to loop over the right-most indexes first.

\subsection{Expressions}
We will work with a subset of all possible \FORTRAN/~77
expressions.\footnote{The entire grammar for \FORTRAN/~77 may be found
at \url{https://slebok.github.io/zoo/fortran/f90/waite-cordy/extracted/index.html}}

\subsubsection{Arithmetic Expressions}
These are built up as:
\begin{enumerate}
\item Primary expressions: unsigned arithmetic constant, identifier of
  an arithmetic constant, arithmetic variable reference, arithmetic
  array element reference, arithmetic function reference, or an
  arithmetic expression in parentheses;
\item Factors, which is either a primary expression or a primary
  expression raised to the power of another factor;
\item Terms: factors, or terms divided or multiplied by factors; 
\item Arithmetic Expressions: a term, a signed term, an arithmetic
  expression plus-or-minus a term.
\end{enumerate}

\subsubsection{Logical expressions}
The literal values ``\texttt{.true.}'' and ``\texttt{.false.}'' are the
only possible literals for logical expressions. The primitive logical
operators, in order of highest-to-lowest precedence:
\begin{itemize}
\item ``\texttt{.NOT.} b'' is the logical negation of $b$;
\item ``a \texttt{.AND.} b'' is the conjunction of $a$ and $b$ \emph{but it is not short-circuiting};
\item ``a \texttt{.OR.} b'' is the disjunction of $a$ or $b$ \emph{but it is not short-circuiting};
\item ``a \texttt{.EQV.} b'' tests $a$ has the same logical value as $b$;
\item ``a \texttt{.NEQV.} b'' tests that $a$ does not have the same
  logical value as $b$.
\end{itemize}
Be careful, none of these are short-circuiting.

We have the following primitive logical operators which act on expressions:
\begin{itemize}
\item ``\texttt{.LT.}'' tests for strictly less than;
\item ``\texttt{.LE.}'' tests for less than or equal to;
\item ``\texttt{.EQ.}'' tests the left-hand side is equal to the
  right-hand side;
\item ``\texttt{.NE.}'' tests the left-hand side is not equal to the
  right-hand side;
\item ``\texttt{.GT.}'' tests for strictly greater than;
\item ``\texttt{.GE.}'' tests for greater than or equal to.
\end{itemize}
When comparing to arithmetic expressions $\ell$ and $r$, it compares
$\ell-r$ to zero.

\subsubsection{Grammar}
The grammar for expressions suitable for a recursive descent parser,
using pidgin EBNF:

\begin{tabular}{rcl}
$\langle$\textit{Expr}$\rangle$ & $::=$ & $\langle$\textit{level 5 expr}$\rangle$\\
$\langle$\textit{level 1 expr}$\rangle$ & $::=$ & $\langle$\textit{primary}$\rangle$\\
$\langle$\textit{primary}$\rangle$ & $::=$ & $\langle$\textit{number}$\rangle$\\
  & $\mid$ & $\langle$\textit{named data reference}$\rangle$\\
  & $\mid$ & $\langle$\textit{function call}$\rangle$\\
\end{tabular}

\begin{tabular}{rcl}
  $\langle$\textit{level 2 expr}$\rangle$ & $::=$ &
  $\langle$\textit{sign}$\rangle^{?}$ $\langle$\textit{add operand}$\rangle$ 
  $\{\langle$\textit{add op}$\rangle$ $\langle$\textit{add operand}$\rangle\}$\\
  $\langle$\textit{sign}$\rangle$ & $::=$ & \texttt{"+"} $\mid$ \texttt{"-"}\\
  $\langle$\textit{add op}$\rangle$ & $::=$ & \texttt{"+"} $\mid$ \texttt{"-"}\\
  $\langle$\textit{add operand}$\rangle$ & $::=$ & $\langle$\textit{mult operand}$\rangle$  $\{\langle$\textit{mult op}$\rangle$ $\langle$\textit{mult operand}$\rangle\}$\\
  $\langle$\textit{mult op}$\rangle$ & $::=$ & \texttt{"*"} $\mid$ \texttt{"/"}\\
  $\langle$\textit{mult operand}$\rangle$ & $::=$ & $\langle$\textit{level 1 expr}$\rangle$ \\
  & $\mid$ & $\langle$\textit{level 1 expr}$\rangle$ \texttt{"**"} $\langle$\textit{mult operand}$\rangle$\\
\end{tabular}

\begin{tabular}{rcl}
  $\langle$\textit{level 3 expr}$\rangle$ & $::=$ & $\langle$\textit{level 2 expr}$\rangle$ $\{$\texttt{"//"} $\langle$\textit{level 2 expr}$\rangle\}$\\
\end{tabular}

\begin{tabular}{rcl}
  $\langle$\textit{level 4 expr}$\rangle$ & $::=$ & $\langle$\textit{level 3 expr}$\rangle$ $\{\langle$\textit{rel. op}$\rangle$ $\langle$\textit{level 3 expr}$\rangle\}$\\
  $\langle$\textit{rel. op}$\rangle$ & $::=$ & \texttt{".EQ."} $\mid$
  \texttt{".NE."} $\mid$
  \texttt{".LE."} $\mid$
   \texttt{".LT."} $\mid$
  \texttt{".GE."} $\mid$
   \texttt{".GT."}\\
\end{tabular}

\begin{tabular}{rcl}
  $\langle$\textit{level 5 expr}$\rangle$ & $::=$ & $\langle$\textit{equiv operand}$\rangle$ $\{\langle$\textit{equiv op}$\rangle$ $\langle$\textit{equiv operand}$\rangle\}$\\
  $\langle$\textit{equiv op}$\rangle$ & $::=$ & \texttt{".EQV."} $\mid$ \texttt{".NEQV."}\\
  $\langle$\textit{equiv operand}$\rangle$ & $::=$ & $\langle$\textit{or operand}$\rangle$
  $\{$\texttt{".OR."} $\langle$\textit{or operand}$\rangle\}$\\
  $\langle$\textit{or operand}$\rangle$ & $::=$ & $\langle$\textit{and operand}$\rangle$
  $\{$\texttt{".AND."} $\langle$\textit{and operand}$\rangle\}$\\
  $\langle$\textit{and operand}$\rangle$ & $::=$ & $\langle$\textit{level 4 expr}$\rangle$\\
  & $\mid$ & \texttt{".NOT."} $\langle$\textit{level 4 expr}$\rangle$\\
  %% & $\mid$ & $\langle$\textit{}$\rangle$\\
  %% $\langle$\textit{}$\rangle$ & $::=$ & $\langle$\textit{}$\rangle$ \\
  %% & $\mid$ & $\langle$\textit{}$\rangle$\\
\end{tabular}

\medbreak
Technically, this breaks compatibility with the Standard, since it will
parse, e.g., \texttt{X + Y + Z} as 
as \texttt{X + (Y + Z)} but the Standard specifically states it should
be equivalent to \texttt{(X + Y) + Z}.

\subsection{Functions}
We can define functions by writing something like

\begin{lstlisting}
      real function sgn(x)
      real x

      sgn = 0
      if (x .GT. 0) sgn = 1
      if (x .LT. 0) sgn = -1

      return
      end
\end{lstlisting}
\FORTRAN/ requires every function have a type, the return value should
be stored in a variable with the same name as the function, and
functions are terminated by the ``\texttt{return}'' statement.

The main program must declare the type of the function, so we would use
this function by writing something like:
\begin{lstlisting}
      program main
      real sgn
C do some calculations here
      stop
      end
\end{lstlisting}

\subsubsection{Subroutines}
Subroutines are similar to functions, except that they do not return a
value, and invoking a subroutine requires different syntax (we write
``\texttt{call subroutine(args...)}''). The semantics are similar,
except that a subroutine call is a statement, whereas a function call
may be an expression or a statement.

\subsubsection{Parameter Passing}
Function calls in \FORTRAN/~77 uses the ``call-by-reference'' strategy
for parameter passing. Effectively, this passes a pointer to the value.
This is specified by \S15.9 of the Standard. The LLVM documentation for
\FORTRAN/ function calling
conventions\footnote{\url{https://releases.llvm.org/11.0.1/tools/flang/docs/Calls.html}}
is worth reading. When passing in a literal value (e.g., in ``\texttt{foo(1)}'')
we need to store the constant, and pass its address to the
function.\footnote{For an enlightening discussion, see \url{https://retrocomputing.stackexchange.com/q/19929}}

% https://mesoscale.agron.iastate.edu/fortran/fort_6.htm
