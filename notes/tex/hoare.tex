\section{Hoare Logic}
Having sketched out informally the syntax and semantics for the fragment
of \FORTRAN/~77 our compiler is targeting, we would like to now focus on
providing a Hoare logic for the fragment of \FORTRAN/. For the most
part, this is straightforward (modulo the subtlety surrounding do-loops
always executing the body of its loop at least once). The only
difficulties we need to address are the \texttt{goto} statement and
working with arrays.

The game plan is to write down the inference rules for the various
statements we will encounter in a \FORTRAN/ program unit.

\subsection{Inference rules}
Statements in \FORTRAN/ can have optional labels (so we can invoke
\texttt{goto}). Hoare logic doesn't handle labels and \texttt{goto}
statements well. Usually the trick is to have the ``context part'' $\Gamma$ of
the judgement $\Gamma\vdash\varphi$ keep track of labels (plus their
preconditions). If $\ell$ is a label, we will write $\Gamma(\ell)$ for
its precondition in the current program unit.

\subsubsection{Continue Statement}
This is literally the ``skip'' statement in Hoare logic. Thus its rule is:
\begin{equation}
\frac{}{\Gamma\vdash\{P\}~\mathtt{continue}~\{P\}}.
\end{equation}

\subsubsection{GOTO Statement}
This is one of the two places where a label is used as part of the
statement. (The other place is the \texttt{do}-loop.)
\begin{equation}
\frac{\Gamma\vdash P\implies\Gamma(\ell)}{\Gamma\vdash\{P\}~\mathtt{goto}~\ell~\{\bot\}}.
\end{equation}
This rule looks weird because control is transferred to the statement
at $\ell$, so the post-condition of the \texttt{goto} statement is never
encountered. The weakest precondition calculus gives us $\bot$ as the
postcondition.

\subsection{Literature review}
Clint and Hoare provided a rule for reasoning about \texttt{goto}
statements. Later, de Bruin~\cite{deBruin1981goto} provided a system
which required working with global invariants for labeled statements.
Boyer and Moore~\cite{boyer1980vcg} have written a verification
condition generator for \FORTRAN/, which has been instructive.
The \textsc{Camfort} libraries\footnote{For example, \url{https://github.com/camfort/fortran-vars}} has produced a static toolkit for \FORTRAN/
(including \FORTRAN/~77).

K. Zimmermann's ``Outline of a formal definition of FORTRAN''~\cite{zimmerman1969outline} is a
tech report lost to the sands of time, but ostensibly formalized a large
fragment of \FORTRAN/~66 in the Vienna Definition Language.

% restricted use of goto:
% Arbib79, Luckham80, Cristian84
% - Arbib, M. A., and S. Alagic. “Proof Rules for gotos" Acta Informatica 11 (1979), 139-148. 
% - Cristian, F. “Correct and Robust Programs.” IEEE Trans. Software Eng. SE-10 (1984), 163-174.
% - Luckham, D. C., and W. Polak. “Ada Exception Handling: An Axiomatic Approach.” ACM Trans.In the functional correctness setting, program correctness is defined as a correspondence between a Prog. Lang. and Syst. 2 (1980), 225-233.

% arrays:
% [Gries81] and [Reynolds81], Luckham79
% - Gries, D. The Science of Programming. New York: Springer-Verlag, 1981. 
% - Luckham, D. C., and N. Suzuki. “Verification of Array, Record, and Pointer Operations in Pascal.” ACM85 Trans. Prog. Lang. and Syst. 1 (1979), 226-244.
% - Reynolds, J. C. The Craft of Programming

