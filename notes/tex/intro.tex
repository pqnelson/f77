\section{Introduction}

\subsection{Why Fortran?}
Fortran is the \emph{lingua franca} for numerical analysis, and
\FORTRAN/~77 is suitably simple enough for writing a compiler. It's a
``blub'' language, ostensibly describable by a suitable semantics. This
gives us a way to verify the implementation (as realized by the
compiler) matches the specification (as described by the Hoare logic).

Arguably, ``old Testament \FORTRAN/'' is close to assembly to a fault.
Without \texttt{goto} and arithmetic-if statements, \FORTRAN/~77 is no
longer Turing complete. It's unclear whether this is significant for
numerical analysis or not, since numerical analysis is so narrowly
focused on particular problems that full Turing completeness may not be
needed. 

\subsection{Outline of paper}
In Section~\ref{section:specification} we describe the subset of
\FORTRAN/~77 supported by our compiler.

In Section~\ref{section:hoare}, we outline the literature on attempts to
formally define ``old Testament \FORTRAN/'' using Hoare logic and
algebraic semantics.
